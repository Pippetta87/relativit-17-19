\documentclass[10pt,xcolor={usenames},fleqn,mathserif,serif]{beamer}


\hypersetup{pdfpagemode=FullScreen}

\addtobeamertemplate{block begin}{%
  \setlength{\textwidth}{0.95\textwidth}%
  \setlength\abovedisplayskip{0pt}%
}{}

\setbeamertemplate{caption}{\insertcaption}

%% colors
\definecolor{bittersweet}{rgb}{1.0, 0.44, 0.37}
\definecolor{brilliantlavender}{rgb}{0.96, 0.73, 1.0}
\definecolor{antiquefuchsia}{rgb}{0.57, 0.36, 0.51}
\definecolor{violetw}{rgb}{0.93, 0.51, 0.93}
\definecolor{Veronica}{rgb}{0.63, 0.36, 0.94}
\definecolor{atomictangerine}{rgb}{1.0, 0.6, 0.4}
\definecolor{darkgray}{rgb}{0.66, 0.66, 0.66}
\definecolor{brightcerulean}{rgb}{0.11, 0.67, 0.84}
\definecolor{cadmiumorange}{rgb}{0.93, 0.53, 0.18}
\definecolor{ochre}{rgb}{0.8, 0.47, 0.13}
\definecolor{midnightblue}{rgb}{0.1, 0.1, 0.44}
\definecolor{lemon}{rgb}{1.0, 0.97, 0.0}
\definecolor{grey}{rgb}{0.7, 0.75, 0.71}
\definecolor{amber}{rgb}{1.0, 0.75, 0.0}
\definecolor{almond}{rgb}{0.94, 0.87, 0.8}
\definecolor{bf}{RGB}{88, 86, 88}
\definecolor{bb}{RGB}{177, 177, 177}


%%%%%%%%%%%%%%%%%%%%%%%%%%%%%%%%%%% importa pacchetti
\usepackage{usepkg}
%%%%%%%%%%%%%%%%%%%%%%%%%%%%%%%%%%% Funzioni generali
\usepackage{functions}
%http://tex.stackexchange.com/questions/246/when-should-i-use-input-vs-include
\newcommand{\setmuskip}[2]{#1=#2\relax} %%problem usinig mu with calc (req by mathtools) loaded
\usepackage{sources}
%\usepackage{length}
%%%%%%%%%%%%%%%%%%%%%%%%%%%%%%%%%%% Funzioni per questo file main
\usepackage{mathOp}

\def\status{ripetere}
\def\keeptrying{coazione}
\usepackage{LocalF}
%%%%%%%%%%%%%%%%%%%%%%%%%%%%%%%%%
\usepackage{beamersetup}
% Let's get started
\begin{document}

\begin{frame}
  \titlepage
\end{frame}

% Section and subsections will appear in the presentation overview
% and table of contents.
%\frame{\tableofcontents[onlyparts]}



%\begin{frame}{Argomenti}
%  \tableofcontents[part=1,hideallsubsections%,pausesections
%  ]
%  % You might wish to add the option [pausesections]
%\end{frame}

\part{relativity blowing}

\begin{frame}[allowframebreaks]{Reg Lez}

\begin{itemize}
  
\item lezione: Principio di relativit\'a. Trasformazioni di Lorentz. Struttura causale dello spazio-tempo. Vettori, covettori e tensori.

\item lezione: Derivate e integrali nello spazio-tempo. Azione della particella libera. Trasformazione delle velocit\'a.

\item lezione: Elettrodinamica classica con il formalismo covariante. Invarianza di gauge e conservazione della carica. Azione per particelle cariche e campi.

\item lezione: Energia ed impulso e tensore associato e sua conservazione. Fluido di polvere e fluido perfetto. Derivazione tensore energia impulso da un'azione.

\item lezione: Tensore energia-impulso per il campo elettromagnetico e sua derivazione a partire dall'azione. Conservazione in presenza di particelle cariche. Esercizio sulla trasformazione dei campi.

\item lezione: Gravit\'a Newtoniana e gravit\'a relativistica. Costanti fondamentali, analisi dimensionale e raggio gravitazionale. Principio di equivalenza debole e forte. Luce in campo gravitazionale. Cambio generale di sistema di coordinate

\item lezione: Definizione e vari esempi di vettori, covettori e tensori. Metrica come campo tensoriale. Tensori invarianti e forma volume. Metriche costanti e statiche.

\item lezione: Derivata covariante. Connessione di Levi-Civita e sue propriet\'a di trasformazione. Esistenza dei sistemi localmente inerziali. Trasporto parallelo.

\item lezione: Metrica costante e shift gravitazionale delle frequenze. Nozione di metrica spaziale. Divergenza covariante e operatore di Lapace-Beltrami. Esercizio sulle coordinate sferiche.

\item lezione: Conseguenze dei principi di equivalenza e covarianza. Equazione delle particelle libere massive in campo gravitazionale e interpretazione come equazione delle geodetiche. Equazioni dell'elettromagnetismo e nozioni elementari sulle forme differenziali. Equazione della luce in spazio curvo.

\item lezione: Trasporto parallelo su circuiti chiusi. Teorema di Stokes e integrali su linee e superfici. Tensore di Riemann, sue propriet\'a di simmetria e numero di gradi di libert\'a in varie dimensioni. Tensore di Ricci e curvatura scalare.

\item lezione: Curvatura dal commutatore delle derive covarianti. Tensore di Weyl. Esercizio sulla metrica sferica e iperbolica. Identit\'a di Bianchi e tensore di Einstein.

\item  lezione: Equazione della deviazione geodetica ed effetto di marea. Esercizi sulla curvatura. Teorema della divergenza, correnti e conservazioni in spazio curvo. Tensore energia-impulso e sua divergenza covariante nulla in spazio curvo.

\item lezione: Derivazione delle equazioni di Einstein. Possibilit\'a del termine di costante cosmologica. Discussione sulla natura delle equazione, loro non linearit\'a e numero di gradi di libert\'a. Prima parte della azione gravitazionale.
 
 \item lezione: Ripasso della derivazione delle equazioni di Einstein. Derivazione del tensore energia-impulso dalla variazione della metrica. Azione di Einstein-Hilbert e termine di costante cosmologica.
 
\item lezione: Metrica di Schwarzschild e teorema di Birkhoff. Moto delle particelle nella metrica di Schwarzschild. Precessione del perielio e deviazione dell'angolo della luce.

\item lezione: Metodo per calcolare costanti del moto e connessione. Discussione dell'energia conservate per le traiettorie nella metrica di Schwarzschild. Discussione degli invarianti di curvatura. Traiettorie che si avvicinano all'orizzonte e redshift gravitazionale.

\item lezione: Singolarit\'a dei sitemi di coordinate ed esempio della metrica di Rindler. Coordinate di Eddington-Finkelstein e diagramma di Finkelstein. Concetto di orizzonte degli eventi e di buco nero. Esercizio precessione perielio. Diffeomorfismi infinitesimi, vettori di Killing ed esempio nel caso di Schwarzschild.

\item  Equilibrio idrostatico per oggetti a simmetria sferica ed equazione di Tolman-Oppenheimer-Volkoff. Esempio di soluzione nel caso di densita' costante. Limite superiore all'esistenza di equilibrio e collasso gravitazionale. Completamento massimale di Schwarschild con le coordinate di Kruskal-Szekeres.

\item lezione: Equazione di Einstein linearizzate per il campo debole. Soluzione delle onde nel vuoto per l'elettromagnetismo e la gravita' debole, gradi di liberta' fisici e polarizzazioni. Problema dell'irraggiamento, soluzione generale, limiti di zona di radiazione e sorgenti non relativistiche.

\item  lezione: Irraggiamento gravitazionale e limite di onda di quadrupolo. Stima ordine di grandezza per una sorgente binaria. Nozione di tensore energia-impuso per il campo gravitazionale ed esempio nel caso di onda gravitazionale.

\item  lezione: Principio cosmologico di omogeneit\'a e isotropia. Spazi massimamente simmetrici in tre dimensioni spaziali. Metrica di Friedmann-Robertson-Walker. Redshift cosmologico delle frequenze, costante di Hubble ed espansione dell'universo.

\item  lezione: Soluzione dell'universo chiuso, aperto o piatto dominato da radiazione o materia. Singolarit\'a del big bang e concetto di orizzonte in cosmologia. Componente di materia barionica e di radiazione cosmica di fondo. Tempi di equivalenza e ricombinazione. Problema dell'orizzonte. 

\item lezione: Universo statico di Einstein. Spazio-tempo massimamente simmetrico e costante cosmologica. Spazio de Sitter. Periodo di inflazione e soluzione del problema dell'orizzonte.

\item lezione: Composizione dell'universo attuale e modello LambdaCDM. Problema della curvatura e soluzione con l'inflazione. Problema della costante cosmologica. 

\end{itemize}

\end{frame}

\part{Problematiche di relativit\'a}\label{part:mainconcepts}
\frame{\partpage}

\begin{frame}[label={intro}]{TOC}

\tableofcontents[onlyparts]

\end{frame}

\begin{wordonframe}{Problematiche relativit\'a}

dove \'e necessari quindi la RG?

\end{wordonframe}



\end{document}
\section{RegLez 18/19}
\begin{frame}[allowframebreaks]{RegLez 18/19}\linkdest{reglez18}
\begin{itemize}
    \item 18/09/2018 lezione: Principio di relativit\'a. Trasformazioni di Lorentz. Struttura causale dello spazio-tempo. Vettori, covettori e tensori. Derivate e integrali nello spazio-tempo. Azione della particella libera.
\item 20/09/2018 lezione: Elettrodinamica classica con il formalismo covariante. Invarianza di gauge e conservazione della carica. Azione per particelle cariche e campi.
\item 25/09/2018 lezione: Energia ed impulso e tensore associato e sua conservazione. Fluido di polvere e fluido perfetto. Derivazione tensore energia impulso da un'azione.
\item 27/09/2018 lezione: Tensore energia-impulso per il campo elettromagnetico e sua derivazione a partire dall'azione. Conservazione in presenza di particelle cariche. Esercizio sulla trasformazione dei campi
\item 02/10/2018 lezione: Gravit\'a Newtoniana e gravit\'a relativistica. Costanti fondamentali, analisi dimensionale e raggio gravitazionale. Principio di equivalenza debole e forte. Luce in campo gravitazionale. Cambio generale di sistema di coordinate.
\item 04/10/2018 lezione: Definizione e vari esempi di vettori, covettori e tensori. Metrica come campo tensoriale. Tensori invarianti e forma volume.
\item 09/10/2018 lezione: Metriche costanti e statiche. Derivata covariante. Connessione di Levi-Civita.
\item 11/10/2018 lezione: Verifiche sperimentali del principio di equivalenza. Importanza per la Relatività Generale. Universalit\'a della caduta dei corpi. Esperimenti di Eotvos e Roll, Krotkov, Dicke. Omogeneit\'a dello spazio. Esperimenti di Pound e Rebka e Vessot. Omogeneità del tempo. Misura della miniera di Oklo. Isotropia dello spazio. Esperimento di Brillet e Hall.
\item 15/10/2018 lezione: Invarianza di Lorentz. Precessione di Thomas e g-2 del muone. Confronto tra dilatazione dei tempi, forza di Lorentz, gamma elettromagnetismo e gamma precessione di Lorentz. Discussione classica della deviazione della luce. Precessione del perielio di Mercurio. Teoria e osservazioni.
\item 16/10/2018 lezione: Connessione e sue propriet\'a di trasformazione. Esistenza de sistemi localmente inerziali. Trasporto parallelo. Divergenza covariante e operatore di Lapace-Beltrami. Esercizio sulle coordinate sferiche.
\item 18/10/2018 lezione: Conseguenze dei principi di equivalenza e covarianza. Equazione delle particelle libere massive in campo gravitazionale e interpretazione come equazione delle geodetiche. Equazioni dell'elettromagnetismo e nozioni elementari sulle forme differenziali. Teorema della divergenza, correnti e conservazioni in spazio curvo. Tensore energia-impulso e sua divergenza covariante nulla in spazio curvo.
\item 23/10/2018 lezione: Trasporto parallelo su circuiti chiusi. Tensore di Riemann, sue propriet\'a di simmetria e numero di gradi di libert\'a in varie dimensioni. Tensore di Ricci e curvatura scalare. Curvatura dal commutatore delle derive covarianti. Esercizio sulla metrica sferica e iperbolica.
\item 25/10/2018 lezione: Tensore di Weyl. Identit\'a di Bianchi e tensore di Einstein. Esercizio sulla curvatura. Derivazione delle equazioni di Einstein. Possibilit\'a del termine di costante cosmologica. Discussione sulla natura delle equazione, loro non linearit\'a e numero di gradi di libert\'a.
\item 30/10/2018 lezione: Azione di Einstein-Hilbert e termine di costante cosmologica. Derivazione del tensore energia-impulso dalla variazione della metrica. Metrica di Schwarzschild e teorema di Birkhoff.
\item 06/11/2018 lezione: Moto delle particelle nella metrica di Schwarzschild. Precessione del perielio e deviazione dell'angolo della luce. Traiettorie che si avvicinano all'orizzonte e redshift gravitazionale.
\item 08/11/2018 lezione: Metodo per calcolare costanti del moto e connessione. Discussione dell'energia conservate per le traiettorie nella metrica di Schwarzschild. Discussione degli invarianti di curvatura. Tensore energia-impulso e sua divergenza covariante nulla in spazio curvo.
\item 13/11/2018 lezione: Singolarit\'a dei sitemi di coordinate ed esempio della metrica di Rindler. Coordinate di Eddington-Finkelstein e diagramma di Finkelstein. Concetto di orizzonte degli eventi e di buco nero. Equilibrio idrostatico per oggetti a simmetria sferica ed equazione di Tolman-Oppenheimer-Volkoff. Esempio di soluzione nel caso di densit\'a costante. Limite superiore all'esistenza di equilibrio e collasso gravitazionale.
\item 15/11/2018 lezione: Completamento massimale di Schwarschild con le coordinate di Kruskal-Szekeres. Equazione di Einstein linearizzate per il campo debole. Soluzione delle onde nel vuoto per l'elettromagnetismo e la gravit\'a debole, gradi di libert\'a fisici e polarizzazioni.
\item 20/11/2018 lezione: Problema dell'irraggiamento, soluzione generale, limiti di zona di radiazione e sorgenti non relativistiche. Irraggiamento gravitazionale e limite di onda di quadrupolo. Stima ordine di grandezza per una sorgente binaria.
\item 22/11/2018 lezione: Nozione di tensore energia-impuso per il campo gravitazionale ed esempio nel caso di onda gravitazionale. Discussione con domande degli studenti.
\item 27/11/2018 lezione: Principio cosmologico di omogeneit\'a e isotropia. Spazi massimamente simmetrici in tre dimensioni spaziali. Metrica di Friedmann-Robertson-Walker. Redshift cosmologico delle frequenze, costante di Hubble ed espansione dell'universo.
\item 29/11/2018 lezione: Soluzione dell'universo chiuso, aperto o piatto dominato da radiazione o materia. Singolarit\'a del big bang e concetto di orizzonte in cosmologia. Componente di materia barionica e di radiazione cosmica di fondo. Tempi di equivalenza e ricombinazione. Problema dell'orizzonte.
\item 04/12/2018 lezione: Universo statico di Einstein. Spazio-tempo massimamente simmetrico e costante cosmologica. Spazio de Sitter. Periodo di inflazione e soluzione del problema dell'orizzonte.
\item 06/12/2018 lezione: Test classici della Relativit\'a Generale. Deviazione della luce, ritardo dell'eco radar, precessione di Mercurio. Cenni al formalismo post-newtoniano parametrizzato. Principio di equivalenza forte. Misure di distanza Terra-Luna. Effetto Lense-Thirring.
\item 10/12/2018 lezione: Onde gravitazionali. Interazione con la materia. Rivelazione con misure di tempo. Oggetti compatti. Origine delle stelle di neutroni e possibile dei buchi neri di massa stellare. Il sistema PSR 1913+16. Propriet\'a post-kepleriane. Irraggiamento. Sorgenti: sistemi binari, analisi al prim'ordine di GW150914. Sorgenti periodiche, collassi stellari, fondo stocastico. Osservazioni recenti: il catalogo dei sistemi binari. GW170817: astronomia multimessaggero, caratteristiche della kilonova risultante.
\item 11/12/2018 lezione: Periodo di inflazione e soluzione del problema dell'orizzonte. Composizione dell'universo attuale e modello LambdaCDM. Problema della curvatura e soluzione con l'inflazione. Problema della costante cosmologica.
\end{itemize}
\end{frame}

\section{RegLez 22/23}
\begin{frame}[allowframebreaks]{Reg Lez 22/23}\linkdest{reglez22}
\begin{itemize}
\item Mar 20/09/2022 lezione: Introduzione del corso. Principio di relativita'. Sistemi di riferimento inerziali. Trasformazioni di Galileo, tempo universale e simultaneita' nella meccanica Galileana. Velocita' della luce, intervallo invariante e quadrivettori. Trasformazioni di Lorentz.
\item 21/09/2022 lezione: Trasformazioni di Lorentz, esempi di rotazioni e boost. Trasformazione di vettori e covettori e loro differenza. Struttura causale dello spazio-tempo, futuro, passato e zona di possibile simultaneita' rispetto ad un evento. Tensori invarianti delta (identita'), eta (metrica) e tensore completamente antisimmetrico epsilon. Azione della particella libera, parametrizzazione temporale e limite non relativistico. Azione della particella libera con parametrizzazione generica e di tempo proprio.
\item 22/09/2022 lezione: Equazione del moto della particella e energia-impulso partendo dall'azione. Particella carica in campo esterno con il formalismo covariante e forza di Lorentz. Correnti conservate ed equazione di continuita' e teorama della divergenza quadri-dimensionale. Termine di interzine scritto con la corrente di carica. Invarianza di gauge e termini scalari e gauge invarianti. Possibili termini per la lagrangiana dell'elettromagnetismo, scalari gauge invarianti e quadratici nei campi. Termine topologico uguale a una derivata totale.
\item 27/09/2022 lezione: Campi vettoriali e derivate, gradiente, divergenza e generalizzazione del rotore. Elettrodinamica classica con il formalismo covariante e azione. Equazioni di Maxwell senza sorgente da divergenza del tensore duale. Equazioni di Maxwell con sorgente derivate dalla'azione come equazioni di Eulero-Lagrange.
\item 28/09/2022 lezione: Energia ed impulso e tensore associato e sua conservazione. Derivazione tensore energia impulso da un'azione. Ridefinizione del tensore energia-impulso che lascia invariato energia-impulso totale. Tensore energia-impulso deve essere simmetrico in una teoria relativistica. Tensore energia-impulso per il campo elettromagnetico con il metodo standard delle particelle e sua derivazione a partire dall'azione aggiungendo il termine compensativo per renderlo simmetrico e gauge invariante.
\item 29/09/2022 lezione: Campo scalare massivo e suo tensore energia-impulso. Discussione su segni e stabilita' dei ermini nell'azione. Discussione sui termini della Lagrangiana di Maxwell, segni e invarianza di gauge. energia e impulso per la particella puntiforme. Azione della particella puntiforme in forma quadratica con la metrica unidimensionale per renderla invariante sotto riparametrizzazioni e caso a massa nulla; discussione sul parametro affine. Forme differenziali e derivata esterna e applicazioni in eletrimagnetismo. Corrente di perticelle e tensore energia impulso per particelle puntiformi. Conservazione del tesore energia-impulto totale campi elettromagnetici piu' particelle.
\item 04/10/2022 lezione: Conservazione delle correnti e teorema della divergenza in formulazione quadridimensionale, in form locale e globale. Discussione sugli integrali di volume 4-dimensionale, di linea 1-dimensionale, e di volume 3-dimensionale. Tensore energia-ipulso per un fluido di particelle non interagenti e per un fluido perfetto. Esercizio sulla conservazione del tensire energia-impulso per campo elettromagnetico accoppiato a particelle cariche.
\item 05/10/2022 lezione: Cambio generale di sistema di coordinate e nozioni di geometria differenziale. Spostamento infinitesimo come vettore e vettore derivata tangente ad una curva nello spazio-tempo. Gradiente di una funzione come covettore. Definizione ed esempi di vettori e spazio tangente, covettori e spazio cotangente. Definizione di tensori. Discussione sui tensori invarianti. Metrica come campo tensoriale (0,2) simmetrico e sue proprieta' di trasformazione. Forma canonica della metrica e segnatura della metrica.
\item 06/10/2022 lezione: Discussione su trasformazione della metrica e conoluce in ogni punto. Determinante della metrica.Tensore antisimmetrico e forma volume. Metrica di Minkowsky in coordinate rotanti. Forme differenziali e derivata esterna. Equazioni dell'elettromagnetismo che non dipendono dalla metrica ma solo dalla derivate esterna.
    \end{itemize}
   \end{frame} 